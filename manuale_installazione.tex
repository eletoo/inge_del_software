Di seguito sono riportate le azioni da eseguire al fine di interagire con l'applicazione. 

\section{Specifiche tecniche}
L'applicazione può essere eseguita sui sistemi operativi Windows, Mac OS e Linux.\newline
Si richiede uno spazio di archiviazione di almeno 1.53 MB per consentire il download del file eseguibile. 
Qualora il file eseguibile debba essere scaricato da remoto è necessaria una connessione internet per poter effettuare questo passaggio; dopodiché non sarà richiesta alcuna connessione per consentire l'utilizzo dell'applicazione.\newline
Si richiede inoltre di avere preinstallato la versione 13.0 di Java JRE (o JDK) o, in alternativa, una versione più recente.

\section{Installazione}
\subsection{Windows}
Una volta installata la versione corretta della JRE, scaricato ed estratto il file eseguibile, individuare e raggiungere la directory in cui esso si trova.

Se il file si presenta automaticamente come eseguibile Java sarà sufficiente un \textit{double-click} sul nome del file per lanciare il comando di \textit{run}. In alternativa cliccare con il tasto destro del mouse sul file e selezionare l'applicazione con cui aprire il file tramite il comando ``Apri con"; selezionare dunque ``OpenJDK Platform binary" (o simili) per eseguire il file.\newline

Qualora non fosse possibile eseguire l'applicazione tramite la modalità appena descritta è possibile recarsi nella directory in cui si trova l'eseguibile, cliccare sulla barra degli indirizzi e scrivere \texttt{cmd}: si aprirà così un terminale associato alla directory corrente. Inserire infine il comando \texttt{java -jar inge\_del\_software.jar} e premere invio per lanciare l'esecuzione.


\subsection{Mac OS}
Una volta installata la versione corretta della JRE, scaricato ed estratto il file eseguibile, individuare e raggiungere la directory in cui esso si trova. Aprire un terminale, inserire il comando \texttt{java -jar inge\_del\_software.jar} e premere invio per lanciare l'esecuzione.

\subsection{Linux}
Una volta installata la versione corretta della JRE, scaricato ed estratto il file eseguibile, individuare e raggiungere la directory in cui esso si trova. Eseguire l’applicativo tramite terminale con il comando \texttt{java -jar inge\_del\_software.jar}