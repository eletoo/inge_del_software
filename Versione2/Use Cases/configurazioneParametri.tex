Nella tabella \ref{tab:Use Case 2.1} è riportato il caso d'uso testuale relativo all'impostazione da parte dell'utente Configuratore dei parametri e delle informazioni di scambio.
Le informazioni vengono salvate in modo permanente, in modo da consentirne la lettura ed eventuale modifica ad accessi successivi.

\begin{table}[!]
    \centering
    \begin{tabular}{l|l}
        \hline
        Nome & Configurazione parametri \\
        \hline
        Attore & Configuratore \\
        \hline
        Scenario principale & 
            \begin{tabular}{l}
                 Precondizione: non sono presenti informazioni di scambio nell'applicazione\\
                 1. L’utente inserisce il parametro “Piazza”, ossia la città in cui avvengono\\ gli scambi; viene segnalato che tale informazione non\\sarà più modificabile in futuro.\\
                 2. L'utente inserisce almeno un luogo in cui tali scambi sono effettuati.\\
                 3. L'utente inserisce almeno un giorno della settimana in cui gli scambi\\possono avere luogo.\\
                 4. L'utente inserisce almeno un intervallo orario entro cui effettuare gli\\scambi.\\
                 5. L'utente inserisce la scadenza, ossia il numero massimo di giorni entro\\cui un fruitore può accettare una
                 proposta di scambio avanzata da un\\altro fruitore.\\
                 6. Viene data conferma all'utente che i dati sono stati salvati correttamente\\
                 Fine                 
            \end{tabular}\\
        \hline
        Scenario alternativo & 
            \begin{tabular}{l}
                Precondizione: sono presenti informazioni di scambio nell'applicazione\\
                1.a. Vengono presentate all'utente le informazioni di scambio attualmente\\presenti\\
                Precondizione: l'utente conferma di voler modificare le informazioni presenti\\
                Torna al punto 2
            \end{tabular}\\
        \hline
        Scenario alternativo & 
            \begin{tabular}{l}
                Precondizione: l'utente sceglie di non modificare le informazioni presenti\\
                2.a.a Viene data conferma all'utente che i dati sono stati salvati\\correttamente\\
                Fine
            \end{tabular}\\
        \hline
    \end{tabular}
    \caption{Caso d'uso: Configurazione parametri}
    \label{tab:Use Case 2.1}
\end{table}