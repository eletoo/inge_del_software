\section{Profilo Fruitore}

\subsection{Registrazione e accesso}
Se l'insieme degli utenti Configuratori non è vuoto l'utente può selezionare dal menu principale l'opzione \texttt{Registrati} al fine di creare un nuovo profilo Fruitore. 

Se l'utente è già munito di credenziali può invece selezionare l'opzione \texttt{Accedi} e inserirle direttamente.

Di seguito si mostra l'interazione dell'utente con l'applicazione per la creazione di un nuovo profilo Fruitore:
\lstinputlisting{IO files/Registrazione fruitore.txt}

Di seguito si mostra l'interazione dell'utente con l'applicazione per l'accesso al proprio profilo Fruitore:
\lstinputlisting{IO files/Accesso fruitore.txt}

\subsection{Visualizzazione contenuto gerarchie e informazioni di scambio}
Di seguito si mostra l'interazione dell'utente con l'applicazione per la visualizzazione del contenuto delle gerarchie e delle informazioni di scambio:
\lstinputlisting{IO files/Visualizzazione info.txt}

\subsection{Visualizzazione offerte personali e per categoria}
Di seguito si mostra l'interazione dell'utente con l'applicazione per la visualizzazione delle offerte personali inserite nell'applicazione:
\lstinputlisting{IO files/Visualizzazione offerte - fruitore.txt}

Di seguito si mostra l'interazione dell'utente con l'applicazione per la visualizzazione delle offerte inserite nell'applicazione filtrate per categoria:
\lstinputlisting{IO files/Visualizzazione offerte per categoria - fruitore.txt} 

\subsection{Creazione offerta}
Di seguito si mostra l'interazione dell'utente con l'applicazione per la creazione di un'offerta di un prodotto da scambiare:
\lstinputlisting{IO files/Creazione offerta.txt}

\subsection{Ritiro offerta}
Di seguito si mostra l'interazione dell'utente con l'applicazione per il ritiro di un'offerta di un prodotto; il prodotto ritirato non potrà più essere selezionato per lo scambio né dall'utente che ne è autore né da altri utenti.
\lstinputlisting{IO files/Ritiro offerta.txt}

\subsection{Creazione proposta di scambio}
Di seguito si mostra l'interazione dell'utente con l'applicazione per la creazione di una proposta di scambio. 
L'utente è tenuto a selezionare (se esistente) un'offerta delle proprie e (se esistente) un'offerta di un altro utente appartenente alla stessa categoria foglia della propria offerta. 
Automaticamente lo stato delle due offerte verrà modificato ed esse terranno traccia l'una dell'altra.
\lstinputlisting{IO files/Creazione proposta scambio.txt}

\subsection{Visualizzazione ultimi messaggi relativi agli scambi}
Se uno scambio è ancora in corso è possibile visualizzare l'ultimo messaggio (ossia l'ultima proposta di appuntamento) introdotto dalla controparte dello scambio.
Di seguito si mostra l'interazione dell'utente con l'applicazione:
\lstinputlisting{IO files/Visualizzazione messaggio.txt}

\subsection{Proposta appuntamento per lo scambio}
Non appena l'utente fruitore effettua l'accesso all'applicazione inserendo le proprie credenziali gli vengono presentati gli elenchi delle nuove proposte di scambio accumulate durante la sua assenza e delle proposte di scambio arretrate (in attesa di un accordo relativamente all'appuntamento per lo scambio). 
L'utente interagisce con l'applicazione nel seguente modo per accettare uno scambio e proporre un nuovo appuntamento:
\lstinputlisting{IO files/Proposta appuntamento.txt}