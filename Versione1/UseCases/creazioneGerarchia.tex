Nella Tabella \ref{tab:Use Case 1.3} è riportato il caso d'uso testuale relativo alla creazione di una nuova gerarchia da parte dell'utente configuratore: l'utente, dopo aver effettuato l'accesso, interagisce con l'applicazione chiedendo di aggiungere nuove categorie oppure salvare la configurazione esistente; in questo caso d'uso, pertanto, sono inclusi i casi d'uso "Aggiunta categoria" e "Salvataggio dati", a cui si demandano i compiti relativi.

\begin{table}[!]
    \centering
     \setlength{\leftmargini}{0.4cm}
    \begin{tabular}{l|l}
        \hline
        Nome & Creazione gerarchia\\
        \hline
        Attore & Configuratore\\
        \hline
        Scenario principale & 
            \begin{tabular}{l}
                1. L'utente inserisce il nome della categoria radice\\
                Precondizione: il nome inserito è unico nell'insieme delle radici delle\\gerarchie\\
                2. L'utente inserisce descrizione della categoria radice ed eventuali\\campi nativi\\
                Precondizione: l'utente chiede di aggiungere un'altra categoria alla\\gerarchia\\
                3. Viene segnalato all'utente che ogni categoria nodo deve avere almeno\\due sottocategorie, poi l'utente seleziona la categoria a cui aggiungere\\la sottocategoria\\
                4. $\langle \langle$include$\rangle \rangle$ Aggiunta categoria\\
                Precondizione: ogni categoria nodo della gerarchia contiene almeno\\due sottocategorie\\
                5. Precondizione: l'utente conferma di voler salvare i dati inseriti\\
                6. $\langle \langle$include$\rangle \rangle$ Salvataggio dati\\
                Fine
            \end{tabular}\\
        \hline
        Scenario alternativo & 
             \begin{tabular}{l}
                Precondizione: il nome inserito non è unico nell'insieme delle radici\\delle gerarchie\\
                2.a. Viene segnalato un errore all'utente\\
                Fine
            \end{tabular}\\
        \hline
        Scenario alternativo & 
             \begin{tabular}{l}
                Precondizione: l'utente non vuole aggiungere un'altra categoria alla\\gerarchia\\
                Torna al punto 5
            \end{tabular}\\
        \hline
        Scenario alternativo & 
             \begin{tabular}{l}
                Precondizione: almeno una categoria nodo non contiene almeno due\\sottocategorie\\
                Torna al punto 3
            \end{tabular}\\
        \hline
        Scenario alternativo & 
             \begin{tabular}{l}
                Precondizione: l'utente conferma di non voler salvare i dati inseriti\\
                Fine
            \end{tabular}\\
        \hline
    \end{tabular}
    \caption{Caso d'uso: Creazione gerarchia}
    \label{tab:Use Case 1.3}
\end{table}