Nella Tabella \ref{tab:Use Case 1.1} è riportato il caso d'uso testuale relativo all'accesso dell'utente configuratore al proprio profilo: l'utente interagisce con l'applicazione inserendo il proprio username e password; in questo caso d'uso, pertanto, è incluso il caso d'uso "Creazione configuratore" che viene invocato nella situazione in cui non esista ancora alcun utente o in cui un utente decida di non voler ritentare l'accesso a seguito dell'inserimento di credenziali errate ma piuttosto di creare un nuovo profilo.

\begin{table}[h]
    \centering
    \begin{tabular}{l|l}
        \hline
        Nome & Accesso configuratore \\
        \hline
        Attore & Configuratore \\
        \hline
        Scenario principale & 
            \begin{tabular}{l}
                Precondizione: esiste almeno un profilo Configuratore già registrato\\
                 1. L'utente inserisce le proprie credenziali\\
                 Postcondizione: le credenziali inserite sono corrette e l'utente ha\\accesso all'applicazione\\
                 Fine
            \end{tabular}\\
        \hline
        Scenario alternativo &
            \begin{tabular}{l}
                Precondizione: non esiste alcun profilo già registrato\\
                1.a. $\langle\langle$include$\rangle\rangle$ Creazione Configuratore\\
                Fine
            \end{tabular}\\
        \hline
        Scenario alternativo & 
            \begin{tabular}{l}
                 Precondizione: le credenziali inserite sono errate\\
                 2. Viene segnalato all'utente un errore\\ 
                Fine
            \end{tabular}\\
        \hline
    \end{tabular}
    \caption{Caso d'uso: Accesso Configuratore}
    \label{tab:Use Case 1.1}
\end{table}