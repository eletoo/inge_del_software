\section{Profilo Configuratore}

\subsection{Registrazione e Accesso}
Qualora non sia ancora stato registrato alcun profilo all'interno dell'applicazione, indipendentemente dal fatto che l'utente selezioni l'opzione \texttt{Accedi} o \texttt{Registrati}, egli viene forzato alla creazione di un profilo \texttt{Configuratore}; è infatti previsto che l'insieme dei Configuratori non sia vuoto.  

Al primo accesso, l'utente seleziona l'opzione \texttt{Registrati}: gli vengono comunicate delle credenziali provvisorie generate casualmente che egli inserisce al fine di effettuare il primo login nell'applicazione. Appena avvenuta la conferma di tali credenziali l'utente è tenuto a inserire un proprio username arbitrario che lo identifichi univocamente (viene segnalato un errore in caso lo username inserito sia già associato a un altro utente) e la propria password personalizzata (di cui viene richiesta conferma). 

Ad accessi successivi l'utente seleziona l'opzione \texttt{Accedi} e inserisce direttamente le proprie credenziali personalizzate in fase di registrazione. \bigskip

Di seguito è mostrata l'interazione dell'utente con l'applicazione per la creazione di un profilo Configuratore:

\lstinputlisting{IO files/Registrazione configuratore.txt}

Di seguito è mostrata l'interazione dell'utente con l'applicazione per l'accesso al proprio profilo (nel caso in cui sia già presente almeno un profilo Configuratore all'interno dell'applicazione):

\lstinputlisting{IO files/Accesso configuratore.txt}
A questo punto l'utente ha accesso all'applicazione e può interagire con il menù principale che gli viene presentato.

\subsection{Creazione gerarchia}
Una volta effettuato l'accesso all'applicazione l'utente può selezionare l'opzione \texttt{Crea una nuova gerarchia} per introdurre le informazioni relative a una nuova gerarchia di prodotti. 

L'interazione può avvenire nel seguente modo:
\lstinputlisting{IO files/Creazione gerarchia.txt}
Tramite questa interazione è stata creata una gerarchia \texttt{Libri} contenente le sottocategorie \texttt{Saggi} e \texttt{Romanzi}, ove la sottocategoria \texttt{Romanzi} contiene le sottocategorie \texttt{Rosa} e \texttt{Gialli}. 

Si possono creare gerarchie con un numero arbitrario di livelli di sottocategorie, purché, come specificato durante l'interazione con l'applicazione, ogni categoria \texttt{Nodo} abbia almeno due sottocategorie.


\subsection{Visualizzazione contenuto gerarchie}
Di seguito si riporta l'interazione necessaria al fine di visualizzare il contenuto delle gerarchie presenti nell'applicazione:
\lstinputlisting{IO files/Visualizzazione gerarchie.txt}

\subsection{Salvataggio dati}
Di seguito si riporta l'interazione necessaria al fine di salvare manualmente i dati introdotti nell'applicazione.

Si noti che il salvataggio del contenuto delle gerarchie deve essere effettuato in fase di creazione della gerarchia stessa, rispondendo nel modo desiderato alla richiesta dell'applicazione \texttt{Salvare la gerarchia creata? [Y/N]}.
\lstinputlisting{IO files/Salvataggio dati.txt}

\subsection{Configurazione informazioni di scambio}
Di seguito si riporta l'interazione necessaria al fine di configurare le informazioni di scambio:
\lstinputlisting{IO files/Impostazione infoscambio.txt}

\subsection{Visualizzazione offerte, offerte in scambio e offerte chiuse}
L'utente Configuratore può scegliere di visualizzare tutte le offerte aperte presenti nell'applicazione nel seguente modo:
\lstinputlisting{IO files/Visualizzazione offerte aperte.txt}

Egli può inoltre visualizzare tutte le offerte in scambio presenti nell'applicazione nel seguente modo:
\lstinputlisting{IO files/Visualizzazione offerte in scambio.txt}

Infine, l'utente può visualizzare le offerte chiuse dagli utenti nel seguente modo:
\lstinputlisting{IO files/Visualizzazione offerte chiuse.txt}

\subsection{Caricamento gerarchie e informazioni di scambio in modalità \textit{batch}}
Per poter effettuare il caricamento delle gerarchie e delle informazioni di scambio in modalità \textit{batch} è necessario l'utilizzo di file con estensione \texttt{.json}. 

L'utente deve recarsi nella directory in cui si trova il file eseguibile dell'applicazione. Se essa è già stata eseguita almeno una volta, avrà creato una cartella \texttt{db} e le sotto-cartelle \texttt{conf} e \texttt{jsonFiles} nella sua stessa directory; in alternativa, l'utente deve creare manualmente la cartella denominata \texttt{db} e le due sotto-cartelle al suo interno.

Nella cartella \texttt{conf} egli deve inserire un solo file di configurazione denominato \texttt{conf.json} per assicurarsi di non generare errori nell'applicazione: essa infatti segnalerà un messaggio di errore qualora individui nella directory più di un file di configurazione. 

Al contrario, nella cartella \texttt{jsonFiles} può inserire un numero arbitrario di file con estensione \texttt{.json}, in quanto nell'applicazione è consentito inserire più di una gerarchia. 

Il file per la configurazione di una gerarchia deve essere caratterizzato dalla seguente struttura:
\lstinputlisting{IO files/Modello file gerarchie.txt}

Il file per la configurazione delle informazioni di scambio deve essere caratterizzato dalla seguente struttura:
\lstinputlisting{IO files/Modello file conf.txt}

L'interazione richiesta con l'applicazione al fine di caricare le informazioni di scambio in modalità \textit{batch} è la seguente:
\lstinputlisting{IO files/Import info scambio.txt}

L'interazione richiesta con l'applicazione al fine di caricare le gerarchie in modalità \textit{batch} è la seguente:
\lstinputlisting{IO files/Import gerarchie.txt}