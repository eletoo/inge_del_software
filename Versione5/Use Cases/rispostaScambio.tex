\begin{table}[h!]
    \centering
     \setlength{\leftmargini}{0.4cm}
    \begin{tabular}{|p{3.5cm}|p{10.5cm}|}
        \hline
        Nome & Risposta scambio\\
        \hline
        Attore & Fruitore\\
        \hline
        Scenario principale & 
            \begin{tabular}{p{10cm}}
                1. Precondizione: l’utente ha almeno una proposta di scambio non ancora scaduta da gestire\\
                2. Viene richiesto all’utente se gestire una proposta di scambio tra quelle in cui è coinvolto\\
                Precondizione: l’utente sceglie di gestire una proposta di scambio\\
                3. L’utente seleziona una proposta di scambio da gestire\\
                Precondizione: non è ancora stato proposto alcun appuntamento relativo alla proposta di scambio selezionata al punto 2\\
                4. $\langle\langle$include$\rangle\rangle$ Proposta appuntamento\\
                5. Le offerte coinvolte nella proposta di scambio passano nello stato \textit{in Scambio}\\
                6. Torna al punto 1\\
            \end{tabular}\\
        \hline
        Scenario alternativo & 
            \begin{tabular}{p{10cm}}
                Precondizione: l’utente non ha alcuna proposta di scambio da gestire\\
                2.a. Le offerte coinvolte in proposte di scambio scadute passano nello stato \textit{Aperta}\\
                Fine                
            \end{tabular}\\
        \hline
        Scenario alternativo & 
            \begin{tabular}{p{10cm}}
                Precondizione: l’utente sceglie di non gestire alcuna proposta di scambio\\
                2.a. Le offerte coinvolte in proposte di scambio scadute passano nello stato \textit{Aperta}\\
                Fine
            \end{tabular}\\
        \hline
        Scenario alternativo & 
            \begin{tabular}{p{10cm}}
                Precondizione: è già stato proposto un appuntamento relativo alla proposta di scambio selezionata al punto 3\\
                Precondizione: l’utente sceglie di accettare l’appuntamento proposto\\
                3.a.a. Le offerte coinvolte nella proposta di scambio passano nello stato \textit{Chiusa}\\
                3.a.b. Torna al punto 1\\
            \end{tabular}\\
        \hline
        Scenario alternativo & 
            \begin{tabular}{p{10cm}}
                Precondizione: l’utente sceglie di rifiutare l’appuntamento proposto\\
                3.b.a. $\langle\langle$include$\rangle\rangle$ Proposta appuntamento\\
                3.b.b. Le offerte coinvolte nella proposta di scambio passano nello stato \textit{in Scambio}\\
                3.b.c. Torna al punto 1\\
            \end{tabular}\\
        \hline
    \end{tabular}
    \caption{Caso d'uso: Risposta Scambio}
    \label{tab:Use Case 5.18}
\end{table}