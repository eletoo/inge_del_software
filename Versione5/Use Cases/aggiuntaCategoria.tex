
\begin{table}[!]
    \centering
    \begin{tabular}{|p{3.5cm}|p{10.5cm}|}
        \hline
        Nome & Aggiunta categoria \\
        \hline
        Attore & Configuratore \\
        \hline
        Scenario principale & 
            \begin{tabular}{p{10cm}}
                 1. L'utente inserisce il nome della categoria da creare\\
                 Precondizione: il nome inserito è unico all'interno della gerarchia di appartenenza della categoria\\
                 2. L'utente inserisce la descrizione facoltativa della categoria. \\
                 3. L'utente inserisce eventuali altri campi nativi e specifica se sono a compilazione obbligatoria o facoltativa\\
                 Postcondizione: viene creata una nuova categoria con le caratteristiche specificate\\
                 Fine
            \end{tabular}\\
        \hline
        Scenario alternativo & 
            \begin{tabular}{p{10cm}}
                Precondizione: il nome inserito non è unico all'interno della gerarchia di appartenenza della categoria\\
                2.a. Viene segnalato all'utente un errore\\
                Fine
            \end{tabular}\\
        \hline
        Scenario alternativo & 
            \begin{tabular}{p{10cm}}
                Precondizione: la creazione della categoria non va a buon fine\\
                Fine
            \end{tabular}\\
        \hline
    \end{tabular}
    \caption{Caso d'uso: Aggiunta categoria}
    \label{tab:Use Case 5.5}
\end{table}