
\begin{table}[!]
    \centering
    \begin{tabular}{|p{3.5cm}|p{10.5cm}|}
        \hline
        Nome & Proposta scambio \\
        \hline
        Attore & Fruitore \\
        \hline
        Scenario principale & 
            \begin{tabular}{p{10cm}}
                Precondizione: esiste almeno una categoria selezionabile all’interno dell’applicazione \\
                Precondizione: sono già state impostate le informazioni di scambio \\
                Precondizione: l’utente fruitore possiede almeno un’offerta aperta \\
                1. L’utente seleziona la propria offerta aperta da scambiare\\
                Precondizione: esiste almeno un’offerta aperta nella stessa categoria foglia di quella selezionata al punto 1 non appartenente all’utente fruitore\\
                2. L’utente seleziona un’offerta aperta nella stessa categoria foglia di quella selezionata al punto 1 che non sia di sua proprietà; la Proposta Aperta relativa all’utente passa nello stato di Offerta Accoppiata, mentre la Proposta Aperta selezionata dall’utente passa nello stato di Offerta Selezionata\\
                Postcondizione: viene creato un oggetto Scambio che tiene traccia dell’associazione tra le due offerte\\
                Fine              
            \end{tabular}\\
        \hline
        Scenario alternativo & 
            \begin{tabular}{p{10cm}}
                 Precondizione: non c’è alcuna categoria selezionabile all’interno dell’applicazione\\
                 1.a. Viene segnalato all’utente un errore\\
                 Fine
            \end{tabular}\\
        \hline
        Scenario alternativo & 
            \begin{tabular}{p{10cm}}
                 Precondizione: non sono ancora state impostate le informazioni di scambio\\
                 1.a. Viene segnalato all’utente un errore\\
                 Fine
            \end{tabular}\\
        \hline
        Scenario alternativo & 
            \begin{tabular}{p{10cm}}
                Precondizione: non esiste alcuna offerta aperta nella stessa categoria foglia di quella selezionata al punto 1 non appartenente all’utente fruitore\\
                 2.a. Viene segnalato all’utente un errore\\
                 Fine
            \end{tabular}\\
        \hline
    \end{tabular}
    \caption{Caso d'uso: Proposta Scambio}
    \label{tab:Use Case 5.16}
\end{table}