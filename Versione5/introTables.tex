Nella Tabella \ref{tab:Use Case 5.1} è riportato il caso d'uso testuale relativo all'accesso dell'utente configuratore al proprio profilo: l'utente interagisce con l'applicazione inserendo il proprio username e password; in questo caso d'uso, pertanto, è incluso il caso d'uso "Creazione configuratore" che viene invocato nella situazione in cui non esista ancora alcun utente o in cui un utente decida di non voler ritentare l'accesso a seguito dell'inserimento di credenziali errate ma piuttosto di creare un nuovo profilo.\bigskip

Nella Tabella \ref{tab:Use Case 5.2} è riportato il caso d'uso testuale relativo alla creazione di un nuovo profilo configuratore: l'utente riceve la comunicazione delle proprie credenziali temporanee, accede al proprio profilo e modifica le proprie credenziali a piacere (rispettando il vincolo relativo all'univocità dello username).\bigskip

Nella Tabella \ref{tab:Use Case 5.3} è riportato il caso d'uso testuale relativo alla creazione di una nuova gerarchia da parte dell'utente configuratore: l'utente, dopo aver effettuato l'accesso, interagisce con l'applicazione chiedendo di aggiungere nuove categorie oppure salvare la configurazione esistente; in questo caso d'uso, pertanto, sono inclusi i casi d'uso "Aggiunta categoria" e "Salvataggio dati", a cui si demandano i compiti relativi.\bigskip

Nella Tabella \ref{tab:Use Case 5.4} è riportato il caso d'uso testuale relativo alla visualizzazione di tutte le gerarchie presenti all'interno dell'applicazione da parte dell'utente configuratore; anche in questo caso d'uso è necessario che l'utente sia autorizzato ad accedere all'applicazione prima di poter interagire avanzando altre richieste.\bigskip

Nella Tabella \ref{tab:Use Case 5.5} è riportato il caso d'uso testuale relativo all'aggiunta di una categoria a una gerarchia; si tratta di un caso d'uso incluso in quello relativo alla creazione di una gerarchia.\bigskip

Nella Tabella \ref{tab:Use Case 5.6} è riportato il caso d'uso testuale relativo al salvataggio dei dati introdotti nell'applicazione dall'utente configuratore; anche in questo caso d'uso è necessario che l'utente sia autorizzato ad accedere all'applicazione prima di poter interagire avanzando altre richieste.\bigskip

Nella tabella \ref{tab:Use Case 5.7} è riportato il caso d'uso testuale relativo all'impostazione da parte dell'utente Configuratore dei parametri e delle informazioni di scambio. Le informazioni vengono salvate in modo permanente, in modo da consentirne la lettura ed eventuale modifica ad accessi successivi.\bigskip

Nella Tabella \ref{tab:Use Case 5.8} è riportato il caso d'uso testuale relativo alla creazione di un nuovo profilo fruitore: l'utente interagisce con l'applicazione inserendo il proprio username e password.\bigskip

Nella Tabella \ref{tab:Use Case 5.9} è riportato il caso d'uso testuale relativo all'accesso dell'utente fruitore al proprio profilo: l'utente interagisce con l'applicazione inserendo il proprio username e password, la cui correttezza viene verificata dall'applicazione.\bigskip

Nella Tabella \ref{tab:Use Case 5.10} è riportato il caso d'uso testuale relativo alla visualizzazione da parte di un utente fruitore delle informazioni di scambio e di nome e descrizione delle categorie radice delle gerarchie presenti nell'applicazione a seguito dell'inserimento da parte degli utenti configuratori.\bigskip

Nella Tabella \ref{tab:Use Case 5.11} è riportato il caso d'uso testuale relativo al ritiro di un'offerta da parte del fruitore: l'utente può scegliere un'offerta aperta e modificarne lo stato ritirandola.\bigskip

Nella Tabella \ref{tab:Use Case 5.12} è riportato il caso d'uso testuale relativo alla creazione di una offerta da parte del fruitore: l'utente dovrà interagire con il sistema compilando i campi necessari per la corretta creazione della suddetta offerta.\bigskip

Nella Tabella \ref{tab:Use Case 5.13} è riportato il caso d'uso testuale relativo alla visualizzazione delle offerte relative ad una specifica categoria foglia da parte di un qualsiasi utente (sia fruitore che configuratore).\bigskip

Nella Tabella \ref{tab:Use Case 5.14} è riportato il caso d'uso testuale relativo alla visualizzazione da parte di un utente fruitore delle proprie offerte di baratto inserite, indipendentemente dal loro stato.\bigskip

Nella Tabella \ref{tab:Use Case 5.15} è riportato il caso d'uso testuale relativo alla visualizzazione da parte di un utente configuratore delle proprie offerte di baratto relative ad una categoria selezionata il cui stato è \textit{Chiusa}.\bigskip

Nella Tabella \ref{tab:Use Case 5.16} è riportato il caso d'uso testuale relativo alla creazione della vera e propria proposta di scambio da parte di un utente fruitore. L'utente seleziona la propria offerta aperta da accoppiare con un'altra offerta aperta nella stessa categoria ma non di sua proprietà.\bigskip

Nella Tabella \ref{tab:Use Case 5.17} è riportato il caso d'uso testuale relativo alla visualizzazione dell'ultimo messaggio inviato dalla controparte coinvolta nello scambio. Il messaggio riguarda un'operazione di scambio prescelta dall'utente.\bigskip

Nella Tabella \ref{tab:Use Case 5.18} è riportato il caso d'uso testuale relativo alla risposta di scambio dell'utente che gestisce una proposta di scambio con un'altro utente.\bigskip

Nella Tabella \ref{tab:Use Case 5.19} è riportato il caso d'uso testuale relativo alla proposta di appuntamento, l'utente in questo caso ha la possibilità di selezionare luogo, giorno e orario per fissare un appuntamento con la controparte coinvolta nello scambio.\bigskip 

Nella Tabella \ref{tab:Use Case 5.20} è riportato il caso d'uso testuale relativo alla visualizzazione delle offerte in scambio, l'utente interagisce con il sistema per richiedere la categoria delle \textit{Offerte in Scambio} di una specifica categoria.\bigskip

Nella Tabella \ref{tab:Use Case 5.21} è riportato il caso d'uso testuale relativo all'importazione delle informazioni da back-end, l'utente interagisce con il sistema per importare gli ingressi relativi a nuove gerarchie con relativi nodi e fogli da un file preesistente.\bigskip